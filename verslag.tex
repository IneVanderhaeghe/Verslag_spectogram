\documentclass[a4paper,kul]{kulakarticle} %options: kul or kulak (default)

\usepackage[utf8]{inputenc}
\usepackage[dutch]{babel}

\date{Academiejaar 2019 -- 2020}
\address{
  Faculteit Industriële Ingenieurswetenschappen \\
  Systeemontwerp met HDL \\
  S. Verslype}
\title{Verslag spectrogram}
\author{Robin Nollet, Sebastian Vantomme, Ine Vanderhaeghe}


\begin{document}
	
\maketitle
	
\begin{center}
	\centering
	\vspace*{\fill}
	\huge
	\textbf{Verslag: bouwen van een spectrogram op een FPGA met behulp van de VHDL-taal}
	\vspace*{\fill}
\end{center}
	
\newpage
	
\tableofcontents

\newpage

\section{Probleemstelling: een audiospectrogram}

Doorheen het semester hebben we ons met ons groepje bezig gehouden met het bouwen van een audiospectrogram.

\subsection{Wat is een spectrogram?}

Een spectrogram kan goed samengevat worden volgens de volgende definitie: het is een visuele represenatie van de energie in elke frequentie uitgezet in de tijd. In de audiowereld worden deze frequenties vaak de \textit{pitch} genoemd van het signaal. % Robin: ik schrijf hier straks nog wat verder aan :)

\section{Oplossingsstructuur}
%Algemeen overzicht.

\subsection{Audio binnentrekken}
%INE

\subsection{Fourier nemen}
%SEBA

\subsection{Grafisch weergeven}
% ROBIN

\section{Implementatie}
%Algemeen overzicht

\subsection{Audio binnentrekken}
% INE

\subsection{Fourier nemen}
% SEBA

\subsection{Grafisch weergeven}
% ROBIN

\section{Besluit}
% Iemand die zich geroepen voelt?

Afsluitende tekst.

\end{document}
